\documentclass[12pt]{article}
\usepackage{amsmath,amssymb,bookmark,parskip,custom}
\usepackage[margin=.8in]{geometry}
\allowdisplaybreaks
\hypersetup{colorlinks,
    citecolor=black,
    filecolor=black,
    linkcolor=black,
    urlcolor=black
}
\setcounter{secnumdepth}{5}

\begin{document}

\title{CS 349 --- User Interfaces}
\author{Kevin James}
\date{\vspace{-2ex}Winter 2015}
\maketitle\HRule

\tableofcontents
\newpage

\section{Vector UI}
We can use multiplicative vectors to translate, shift, or rotate our objects. The important vectors are as follows:
\begin{itemize}
\item the {\bf translation matrix} is $\begin{bmatrix}1 & 0 & t_x \\ 0 & 1 & t_y \\ 0 & 0 & 1 \end{bmatrix}$
\item the {\bf scaling matrix} is $\begin{bmatrix}s_x & 0 & 0 \\ 0 & s_y & 0 \\ 0 & 0 & 1 \end{bmatrix}$
\item the {\bf rotation matrix} is $\begin{bmatrix}\cos\Theta & \sin\Theta & 0 \\ \sin\Theta & \cos\Theta & 0 \\ 0 & 0 & 1 \end{bmatrix}$
\end{itemize}

\section{Widgets}
A {\bf widget} is a generic name for a part of an interface with its own behaviour. They tend to have their own appearance, their own purpose, and can be pretty much anything (ie. a scrollbar, a button, a textbox...). A {\bf logical input device} is a graphical component defined by a function rather than by what it looks like. Each devices transmits a set of primitives:
\begin{itemize}
\item locator: an (x,y)-position
\item pick: identifies a displayed object
\item choice: selects from a set of alternatives
\item valuator: inputs a value
\item string: inputs a string of characters
\item stroke: inputs a sequence of positions
\end{itemize}

The primitives are abstracted away so that widgets do not need to handle multiple inputs types, eg. keyboards and voice-recognition.

A widget may be considered a logical input device with an appearance. A logical \emph{button} device can generate a ``pushed'' event, though it may look like a button, be a simple keyboard shortcut, etc.

There exist three types of wisgets: simple, container, and abstract model widgets. We can further characterize them by the model they manipulate, the events they generate, and the properties affecting their appearance.

Goals of a widget toolkit:
\begin{itemize}
\item complete: GUI designers should have everything they need
\item consistent: user sees a consistent look and feel and developers have constistent usage paradigms
\item customizable: developer can reasonably extend functionality to meet the needs of an application
\end{itemize}



\end{document}
