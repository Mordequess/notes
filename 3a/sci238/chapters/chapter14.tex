\section{Chapter 14}
\subsection{Milky Way}
\subsubsection{Appearance}
The Milky Way Galaxy has over 100 billion stars. It is a vast \textbf{spiral galaxy} consisting of spiral arms in a flat disc converging at a bugle in the center. The disc is surrounded by a dimmer halo (it is dimmer because most of the very bright stars are in the disc). The galaxy is 100 000 light years in diameter and 1000 light years thick. Our solar system is about 27 000 light years from the center.

It is difficult to view the galaxy because of clouds of interstellar gas and dust (called the \textbf{interstellar medium}) get in the way.

The Milky Way is one of the larger galaxies in our Local Group and its gravity influences smaller galaxies in the area. The Small and Large Magellanic Cloud galaxies actually orbit the Milky Way. Two even smaller and closer galaxies (Canis Major and Sagittarius Dwarf) are in the process of colliding with the Milky Way which will rip them apart.

\subsubsection{Stars in Orbit}
Stars in the disc of the galaxy orbit in roughly circular paths in the same direction and roughly the same plane. They orbit a bit like marry-go-rounds where the stars orbit the center but also bob up and down as they do. This bobbing happens because of localized gravity within the disc. When a star is to high the disk pulls it downwards, but it overshoots and becomes too low, repeat. The stars at the edge of the galaxy orbit at roughly the same speed as the stars at the center of the galaxy which is what gives it that swirl look.

Stars in the bulge and halo have randomly oriented orbits. Bulge stars move around the galactic center in elliptical paths with random orientations. Halo stars have much more exaggerated orbits, swooping high and low the disc at such high velocities that the disc's gravity barely effects them.

The Orbital Velocity Law:\\
\begin{align*}
M_r = \frac{r \times v^2}{G}
\end{align*}
Where $M_r$ is the amount of mass contained with this orbit (kg), r is the radius of the orbit(m), v is the object's orbital velocity (m/s) and G is the gravitational constant ($6.67\times 10^{-11} \frac{m^3}{kg \times s^2}$).

\subsection{Galactic Recycling}
Interstellar mass is recycled within the galaxy's interstellar medium. This also changes the composition of the medium (stars make much heavier elements in their deaths).

\subsubsection{Gas Recycling}

\begin{itemize}
\item \textbf{atomic hydrogen clouds} - interstellar gas clouds fill the galactic disk
\item \textbf{molecular clouds} - gas in the disk gradually cools and forms molecules
\item \textbf{star formation} - gravity makes stars form molecular hydrogen mass
\item \textbf{nuclear fusion in stars} - fusion in the cores of stars makes new elements from hydrogen
\item \textbf{hot bubbles} - supernovae and stellar winds return gas and new elements to interstellar space
\begin{itemize}
\item the strong solar winds from supernovae sweep surrounding material into a hot bubble, these continue to expand until the breach the galactic disk where they erupt, the erupted gas cools and rains back down onto the disk
\item supernovae also create shock waves that create walls of fast moving gas that heats and ionizes interstellar gas
\end{itemize}
\item \textbf{returning gas} - returning gas cools and then blends into atomic hydrogen clouds
\begin{itemize}
\item longest stage so its where most of the hydrogen lives
\item the matter rained down onto the disk cools and condenses into clouds (these may contain dust grains of carbon or silicon which are what blocks our view)
\item as the cloud cools hydrogen combine into molecules making it a molecular cloud
\end{itemize}
\item repeat
\begin{itemize}
\item in molecular clouds stars are formed
\item these stars' solar wind erode the clouds and keep more stars from forming
\item molecules fall apart and become ionized and join the near by atomic hydrogen clouds
\end{itemize}
\end{itemize}

Different parts of the Milky Way are at difference stages of the cycle so we can view the whole cycle via different wavelengths.

\begin{itemize}
\item radio emissions show atomic hydrogen (has a 21cm spectral line)
\item radio emissions of carbon monoxide show the distribution of molecular clouds
\item long-wavelength infrared emissions from interstellar dust show molecular clouds where stars are forming
\item short-wavelength infrared emissions show the light from stars
\item visible light shows how the galaxy looks and where dust blocks our view
\item X-rays show where hot gas bubbles are
\item gamma-rays show where gas densities are highest (most number of collisions) in molecular clouds
\end{itemize}

\subsection{Location of Star Formation}
Stars form in molecular clouds, duh. Locations where there are hot massive stars are indication of star birth places (those stars dont live very long). These areas also tend to be very colorful due to wisps of hot gas called \textbf{ionized nebulae}. These tend to be redder because electrons falling a level in hydrogen give off red photons. Nebulae that are bluer tend to be that color from dust grains reflecting light.

The spiral arms of our galaxy are full of new stars since they house many mollecular clouds and lots of young bright stars. Spiral arms form because molecular clouds collide quite often, these continue to compact until they become a birthing zone for stars.

\subsection{History of the Milky Way}
Unlike the disk that has stars of all ages, the halo has only old stars that contain fewer heavy elements. This is because the halo does not contain molecular clouds used for star formation so there can be no new stars in the halo and the halo is the oldest part of the galaxy before heavier elements existed as much.

Our galaxy started as a \textbf{protogalacic cloud} containing tones of hydrogen and helium. Gravity causes the cloud to contract and fragment. The bulge and halo stars formed first. At this point the galaxy was not disk shaped so these new stars orbited however they wanted. The gas continued to contract until it flattened into a disk due to the conservation of angular momentum. The concentrations of heavy elements in the halo stars implies that the Milky Way formed in multiple clouds that collided. Some halo stars move in organizes streams that implies that they came from other galaxies that collided with ours.

\subsection{Galactic Center}
The galactic center lies in the direction of Sagittarius. When we look at within 1000 light year of the center of our galaxy we see very dense cloud of gas and several million stars. At the center of this is the source a bright radio emissions, Sagittarius A*. Several hundred stars cluster Sagittarius A* within a light year and their orbital paths indicate a massive object at the center, about 4 million solar masses with a volume smaller than our solar system. This must be a black hole.
