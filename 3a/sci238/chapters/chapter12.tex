\section{Chapter 12 -- Star Life Cycle}
All elements heavier than hydrogen and helium were created through fusion or supernova

\subsection{Star Birth}
gravity causes a gas cloud to contract until center is hot enough to sustain -fusion
heat generated from gravitational potential
clouds internal gas pressure resists gravity

gravitational equilibrium: gas pulling inward matches pressure pushing out, star size is stable

two factors:
    1. higher density, more material per space (still low enough to be strong vacuum on earth)
    2. lower temperature, reduces pressure (typically 10-30 K)

$M_min = 18 * M_Sun * sqrt(T^3 / n)$

T = temp of gas in K
n = density of gas in particles per cubic meter

referred to as molecular clouds, cold enough for H atoms to pair up
very large, many stars generally born in simulatneous clusters


\subsubsection{Protostar Stage}
    as it compresses gas starts to heat up, lumpy clumpy shape
    energy radiates away until its dense enough to trap heat, temp rises hard
    still not hot enough for fusion
    rotate rapidly from conservation of momentum of particles
    flattens to a disk from particle collisions (plantes may form here eventually)
    may shoot jets perpindicular to disk (not sure why, suspect magnetic fields due to rotation)
    field also generates protostellar wind (stronger version of solar wind)
    wind and jets shed momentum by expelling material, slowing rotation

    angular momentum also cause of binary star systems (stars form close together, orbit instead of crash, more momentum = larger orbit)

    becomes true star at 10 million K, continues to rise until balance achieved (energy from fusion = energy radiated)
    time to reach main sequence phase proportional to mass, large stars are faster

stars range in size, over 99\% are within 0.5 and 2 M\_Sun (leaning below 1)
large ones burn out faster

stars cant be more than 300 M\_Sun because it would blow off its outer layers
cant be less than 0.08 M\_Sun or it wont get hot enough, stabilizes as a brown dwarf
brown dwarf gravity collapse halted due to degeneracy pressure, restriction on how close elections can be together

\subsection{Low Mass Stars ($<8$ M\_sun)}
    Main-Sequence stage
        90\% of star's lifetime
        star regulates itself: if fusion works too fast, core expands until it cools again

    \subsubsection{Red Giant stage}
        when core hydrogen is depleted, fusion will cease
        no more radiation pushing outwards, core shrinks from gravity
        core is inert helium, small shell of hydrogen around it fuses (higher rate than core), outer layers expand
        star is 100 times larger and 1000 times brighter than main sequence stage
        weaker gravity at surface, increased stellar wind
        fusion shell makes more helium, core gets heavier and shrinks more, shell gets even hotter and denser

    \subsubsection{Helium Core Fusion stage}
        feedback loop until core reaches 100 million K, helium start fusing into carbon
        at this point thermal pressure is too low (gravity is fucking intense)
        core sustained by degeneracy pressure, which does not increase with temperature
        helium fusion heats the core without causing it to inflate, fusion rate spikes (called helium flash)
        so much so that thermal pressure becomes dominant and core increases in size, lowering temp and fusion
        outer layers shrink again, stabilizes back at yellow
        this stage is short, 1\% of star lifetime

    \subsubsection{Last Gasps}
        when helium runs out carbon core shrinks again
        outer layer expansion again from helium shell fusion (hydrogen shell still going, core double layered)
        now even larger than red giant stage
        star is too low mass to fuse carbon, will not reach 600 million K
        too large  for its mass, gravity too low on surface, outer layers start being blown off
        forms planetary nebula (nothing to do with planets), bright glowing ring
        will combine into interstellar dust when cooled
        exposed core remains as a stable white dwarf, gas recycled into a new star
        will cool until it no longer emits light, then sit in the dark of space


\subsection{High Mass Stars}
    \subsubsection{Hydrogen Fusion}
        once in main stage, protons can slam into carbon, nitrogen, oxygen molecules with enough energy
        follows CNO cycle of fusion and decay
            1. C12 + H -> N13
            2. N13 -> C13
            3. C13 + H -> N14
            4. N14 + H -> O15
            5. O15 -> N15
            6. N15 + H -> C12 + He4

        this cycle allows hydrogen fusion to proceed much faster than proceed much faster than typical proton chain (more valid things to bump into)
        makes these stars much brighter, lives much shorter

    \subsubsection{Becoming a Supergiant}
        reaches hydrogen fusing shell stage much faster, outerlayers expand
        temperatures so high that degeneracy pressure never takes over, no helium flash (gradual, like hydrogen was)
        fuses helium into inert carbon core in just a few thousand years
        core fusion stops, core shrinks, helium shell forms, surface expands
        alternates between shrinking and expanding as core reaches next level of fusion
        hydrogen > helium > carbon > oxygen > neon > magnesium > silicon > iron
        the biggest stars transitions so quickly other layers dont have time to resond, become red supergiant

        eg Betelgeuese, Orion's left shoulder. 500 solar radii, 2 AU

    \subsubsection{Heavier Nuclei}
        simplest heavy fusion is helium-capture reaction
            1. C12 + He4 -> O16
            2. O16 + He4 -> Ne20
            3. Ne20 + He4 -> Mg24

        note that each transition upwards drains the core and causes another shell to form
        all shells will be active simultaneously
        once hot enough, can start fusing those heavy nuclei
            1. C12 + O16 -> Si28
            2. O16 + O16 -> Si31 + H
            3. Si28 + Si28 -> Fe56

    \subsubsection{Iron, the Dead End}
        iron is the only element where it is not possible to generate nuclear energy, fusion or fission
        lowest mass per nuclear particle of all elements
        iron core can only resist gravity through degeneracy pressure, but more iron keeps piling on
        then gravity pushes past the quantum mechanical limit

    \subsubsection{Supernova}
        electrons disappear by combining with protons to form neutrons, realeasing neutrinos
        iron core with a mass near M\_Sun and radius larger than Earth collapses into a ball of neutrons just a few kilometers across in a fraction of a second
        stops due to neuton denegeneracy pressure
        this neutron star is similar to an atom nuecleus the size of Kitchener
        the gravitational collapse of the core releases an enormous amount of energy, more than 100 times than the Sun will radiate over its entire 10 billion year lifetime
        old theory was that supernova was caused by matter collapsing into neutron star and bouncing
        new theory is collapse causes so many neutrinos to be formed that, despite how rarely they interact with matter, entire star is blown away
        so hot that it's as bright as moderately sized galaxies for a few weeks, continue to expand and cool
        will eventually be incorporated into new stars in other gas clouds

    Crab nebula is remnant of supernova from 1054 AD

    if gravity is still strong enough to overcome neutron degeneracy pressure, collapses continues further to a black hole

    interesting note: due to larger stars dying and adding heavier elementss to the interstellar dust, newer stars have higher percentages of heavy elements than older stars (2-3\% vs 0.1\%)

    interesting note: most heavy elements are made in helium capture which adds two protons, so even numbered elements are more abundant in the universe


\subsection{Binary Systems}
    the two stars exert tidal forces on eachother, create football shapes
    when the more massive star begins to expand, the gas on the surface experiences strong pull to the other star than its own core, begins a mass exchange
    may transfer back when "soon to be as or more massive" star begins expanding too
