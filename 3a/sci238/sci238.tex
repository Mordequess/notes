\documentclass[12pt]{article}
\usepackage{amsmath,amssymb,bookmark,parskip,custom}
\usepackage[margin=.8in]{geometry}
\allowdisplaybreaks
\hypersetup{colorlinks,
    citecolor=black,
    filecolor=black,
    linkcolor=black,
    urlcolor=black
}
\setcounter{secnumdepth}{5}

\begin{document}

\title{SCI 238 --- Introduction to Astronomy}
\author{Kevin James}
\date{\vspace{-2ex}Winter 2015}
\maketitle\HRule

\tableofcontents
\newpage

\section{Chapter 1 -- Our Place in the Universe}
\subsection{Overview}
A naive look at the sky, which seems to rotate around us, implies we live in a {\bf geocentric} Universe, ie. that everything orbits around the Earth. We know now that this is untrue, but the path to this knowledge was a long one.

We can refer to our place in the Universe as our {\bf cosmic address}, this is our {\bf solar system}; which consists of the Sun and all objects that orbit it including rocky {\bf asteroids} and icy {\bf comets}. Our solar system, and all the stars we can see, make up a small portion of the {\bf Milky Way} galaxy.

A {\bf galaxy} is an island of stars in space containing anywhere between a few hundred million to trillions of stars. The Milky Way is a relatively large one, with 100 billion stars. We are located about halfway from the center of the Milky Way (the {\bf galactic center}) to the edge of the {\bf galactic disk}.

In summary: the Earth is a planet in the solar system, which is a collection of objects orbiting a star, which is in the milky way galaxy, which is a part of the {\bf local group} of galaxies, which is part of the {\bf local supercluster} of groups, which is somewhere in the {\bf Universe}.

The local group contains about 40 galaxies, and is one of what we call {\bf galactic clusters} -- groups of galaxies with more than a few members. Superclusters are essentially clusters of galactic clusters, as the Universe seems to be arranged ingiant chains and sheets with large divides between them. The local group is on the edge of the local supercluster.

\subsection{The Big Bang and the Expanding Universe}
We have observed that the Universe seems to be \emph{expanding}, that is, the distance between galaxies is increasing. By extrapolating backwards, we imagine that all matter must ahve existed at the same point in the past and exploded outward in a {\bf Big Bang}. Based on the rate of expansion, we believe this happened approximately 14 billion years ago.

Note that though the distance between galaxies is increasing, the distances between objects within galaxies is \emph{not}.

Most galaxies, including our own, formed within a few billion years of the Big Bang.

\subsection{The Birth of Stars}
A star is {\bf born} when gravity compresses the material in a cloud of gas and dust until it is dense and hot enough to generate energy through {\bf nuclear fusion}. The star lives so long as it has useable material to fuel its fusion and dies once it runs out.

A star {\bf dies} by blowing much of its remaining content back into space in a {\bf supernova}. This matter eventually becomes new stars and planets.

\subsection{Modelling the Universe}
\subsubsection{Distance}
We can imagine our solar system shrunk down to a manageable size. On the {\bf Voyage} scale (a model solar system in Washington, D.C. where sizes are one billionth of the actual size), the Sun is roughly the size of a grapefruit, Jupiter the size of a marble, Earth the size of a pinhead. Obviously the Sun is far larger than any planet, in fact, it has more than 1000 times as much mass as all the planets in our system combined. Note that the planets also vary in size to a large extent: the ``permanent'' storm on Jupiter known as the {\bf Giant Red Spot} is larger than the Earth.

We can also use this model to describe the distances between objects. On the same scale, the Earth is 15m from the Sun and the distance from the Sun to pluto is 600m. To have enough space for the orbit of each object in the system, we would need a grid of 300 football fields centered around the Sun.

The closest star (Alpha Centauri) is incredibly far away, on this scale it would be the difference between Washington, D.C. and California.

If we reduce this scale by another factor of one billion, we can start thinking about the size of the galaxy. On this scale, each light year is roughly a millimeter and the Milky Way is about the size of a football field. The distance between our star system and Alpha Centauri is smaller than the width of our pinky.

\subsubsection{Time}
We can also model the time between events by creating a {\bf cosmic calendar}. If we set the Big Bang as January $1^{st}$ and the present day as Decembre $31^{st}$, we can place each major event on specific days. By the age of the universe, each month represents just over a billion years.

On this scale, the Milky Way was formed sometime in February. Our solar system, though was, only formed in September -- life on Earth flourished by the end of that month. Recognizeable animals only appeared in mid-December. Dinosaurs first appeared the day after Christmas and died off yesterday. Around 9PM today, early hominids began to walk upright.

The entire history of human civilization, then, occurred in the last half-minute of January $31^{st}$.The Egyptians built the pyramids in 11 seconds, Galileo proved the Earth orbited the Sun one second ago, and the average college student was born 0.05 seconds ago.

\subsection{Motion of the Universe}
The Earth has a daily rotation (its {\bf spin}) and a yearly {\bf orbit} (or {\bf revolution}) around the Sun.

The Earth rotates each day around its {\bf axis}, the imaginary line from the North to the South pole. It rotates from West to East -- counter-clockwise when viewed from above the North pole. The speed is substantial; anywhere other than near the axes, an object whirls around the Earth at a speed greater than 1000km/h.

The Earth's average orbital distance is one {\bf Astronomical Unit} (AU), which is approximately 150 million kilometers. At times, we race around the Sun in excess of one hundred thousand kilometers per hour.

Earth's orbital path defines the {\bf eliptic plane}, and its axis is tilted by 23.5 degrees from a line perpendicular to this plane. The {\bf axis tilt} happens to align our north pole with Polaris, the North Star.

Note that the Earth orbits the Sun in the same direction as it spins around its axis since it was formed from a spinning disk and both of these rotations are a remnant of this.

We also move relative to nearby stars at a rate of $70,000$ kilometers per hour. We rotate around the Milky Way's galactic center once every 230 million years, which implies speeds of (on average) $800,000$ kilometers per hour.

\subsection{Dark Matter and Dark Energy}
Since stars at different distance from the galactic center orbit at different speeds, we can learn how mass is distributed in the galaxy by measuring the differing speeds. Studies show that the mass of the stars in the galactic disk form only a small percentage of the total mass of the galaxy. Most of the galactic mass, then, seems to be located outside of the visible disk in the galaxy's {\bf halo}. We call this mass {\bf dark matter}, since we have not observed light being emitted from it. Similarly, we find that the bulk of the energy in the universe is {\bf dark energy}. This seems to be the case in all observable galaxies.

\subsection{Relative Galactic Movements}
Within the local group, some galaxies move toward us and some move away. Outside of the local group, though, this changes; virtually every galaxy outside of the local group seems to be racing away from us at a speed proportional to its distance from us. This is the basis of our theory that of an {\bf expanding universe}: that the space between galaxies is increasing.

Note that we observe this motion by measuring {\bf Doppler shifts}, as detecting any difference in celestial position within our lifetimes is impossible.


\section{Definitions}
\subsection{Basic Astronomical Objects}
\begin{itemize}
\item a {\bf star} is a large, glowing ball of gas that generates heat and light through nuclear fusion in its core. Our Sun is a star.
\item a {\bf planet} is a moderately large object that orbits a star and shines primarily by reflecting light from its star. According to a definition
approved in 2006, an object can be considered a planet only if it (1) orbits a star; (2) is large enough for its own gravity to make
it round; and (3) has cleared most other objects from its orbital path. An object that meets the first two criteria but has not
cleared its orbital path, like Pluto, is designated a dwarf planet.
\item a {\bf moon (or satellite)} is an object that orbits a planet. The term satellite can refer to any object orbiting another object.
asteroid A relatively small and rocky object that orbits a star.
\item a {\bf comet} is a relatively small and ice-rich object that orbits a star.
\end{itemize}

\subsection{Collections of Astronomical Objects}
\begin{itemize}
\item a {\bf solar system} is the Sun and all the material that orbits it, including the planets, dwarf planets, and small solar system bodies. Although the term solar system technically refers only to our own star system (solar means “of the Sun”), it is often applied to other star systems as well.
\item a {\bf star system} is a star (sometimes more than one star) and any planets and other materials that orbit it.
\item a {\bf galaxy} is a great island of stars in space, containing from a few hundred million to a trillion or more stars, all held together by gravity and
orbiting a common center.
\item a {\bf cluster (or group) of galaxies} is a collection of galaxies bound together by gravity. Small collections (up to a few dozen galaxies)
are generally called groups, while larger collections are called clusters.
\item a {\bf supercluster} is a gigantic region of space where many individual galaxies and many groups and clusters of galaxies are packed more
closely together than elsewhere in the universe.
\item the {\bf universe (or cosmos)} are the sum total of all matter and energy -- that is, all galaxies and everything between them.
observable universe The portion of the entire universe that can be seen from Earth, at least in principle. The observable universe is
probably only a tiny portion of the entire universe.
\end{itemize}

\subsection{Astronomical Distance Units}
\begin{itemize}
\item an {\bf astronomical unit (AU)} is the average distance between Earth and the Sun, which is about 150 million kilometers. More technically,
1 AU is the length of the semimajor axis of Earth’s orbit.
\item a {\bf light-year} is the distance that light can travel in 1 year, which is about 9.46 trillion kilometers.
\end{itemize}

\subsection{Terms Relating to Motion}
\begin{itemize}
\item {\bf rotation} is the spinning of an object around its axis. For example, Earth rotates once each day around its axis, which is an imaginary
line connecting the North Pole to the South Pole.
\item an {\bf orbit (revolution)} is the orbital motion of one object around another. For example, Earth orbits around the Sun once each year.
\item the {\bf expansion (of the universe)} is the increase in the average distance between galaxies as time progresses. Note that while the universe as a whole is expanding, individual galaxies and galaxy clusters do not expand.
\end{itemize}


\section{Formulae and Values}
The {\bf speed of light} is approximately $300,000$ km/h. A {\bf light-year} is the distance light can travel in one year: $9.46 * 10^{12}$ km, or roughly 10 trillion. Note that based on the speed of light, and since the universe formed roughly 14 billion years ago, the distance of 14 billion light-years forms the boundary of the {\bf observable universe}. Note that this does not place a size limit on the entire universe, only on the portion that we can (and will ever be able to) see.

Our solar system was formed 4.5 billion years ago, when about $2\%$ of the galaxy's original Hydrogen and Helium had been converted to heavier elements. Thus the cloud which formed our galaxy was roughly $98\%$ Hydrogen and Helium. The $2\%$ of other materials form the core of the rocky planets in our systems, ie. the Earth.

The {\bf Andromeda galaxy} is roughly 2.5 million light-years away and about $100,000$ light-years in diameter. {\bf Sirius}, the brightest star visible in the night sky, is 8 light-years away. {\bf Alpha Centauri}, the closest star system to our own (a three star system), is 4.4 light-years away.

\end{document}
