\documentclass[12pt]{article}
\usepackage{amsmath,amssymb,parskip,custom}
\usepackage[margin=1in]{geometry}

\begin{document}

\title{ECE 106 - Electricity and Magnetism}
\author{Kevin Carruthers}
\date{\vspace{-2ex}Winter 2013}
\maketitle\HRule

\section*{Structure of an Atom}
An atom has a nucleus comprised of protons (p+) and neutrons (n). It is surrounded by an electron (e-) cloud. Each has a charge of $\pm 1.6*10^{-19}$ C (note that a single Coloumb would be fucking huge.). Any atom with an equal amount of electrons and protons is electrically neutral. If one or more electrons is removed, the atom is positively charged. Extra electron(s) make the atom negatively charged.

In a {\bf conductor}, the electrons' bond to the nucleus is very weak. This means the electrons are free to move, and thus we say the conductor has a low internal resistance. By bringing a charged object nearby, we can induce a charge into such a medium.

In an {\bf insulator}, the electrons are not free to move; however, the atoms can be slightly "reshaped" internally. An insulator has a high internal resistance. We can also induce charge in inductors: this is known as polarizability.

\section*{Electromagnetic Force}
Electromagnetic force is the second strongest of the four fundamental forces, weaker only than the strong force, which has extremely short range. The electromagnetic force has infinite range, as it is inversely proportional to the distance between interacting charges; it may grow extremely weak, but will never lose all effect.

It can also both attract and repel, depending on the relative differences (from zero) of interacting charges.

\subsection*{Electric Charge}
Electric charge is the fundamental unit upon which electromagnetism acts, much like mass/energy is for gravity. Charge is discrete in nature, which means all values of charge are some integer multiple of the charge of an electron: electromagnetism's fundamental unit. Note that charge can thus be negative or positive, and that opposite charges exert an attractive force. Conversely, like charges repel.

Charge is conserved - it can neither be created nor destroyed, and thus in any isolated (closed) system, the total charge can  never change.

\subsection*{Electric Field}
The electric field $E$ is defined as the force created by a charge, divided by the charge of the interacting particle \[ E = \frac{kq}{d^2}\vec{d} \] (I'll define $k$ soon). Note that positive charges will create a field pointing away from them, and vice versa, as evidenced by the $\vec{d}$. In the future, I will not show this {\bf direction vector} unless necessary for clarity, but it is present in virtually all calculations.

For any charge $q$ in an electric field, we can calculate the {\bf electric force} acting on it by \[ F = qE \] This gives us the general equation for electric force \[ F = \frac{kq_0q_1}{d^2} \] Note the similarity to the equation for gravitational force.

\subsubsection*{Vector Diagrams of the Electric Field}
When sketching fields, the direction of arrows shows the direction of the field, and the density of arrows shows the relative strength of the field (qualitatively). Since fields are vectors, we can then add multiple fields through vectorial addition (and/or superposition).

\section*{Coloumb's Law}
The force acting between two charges is equal to \[ k\frac{q_0q_1}{d^2} \] where $k$ is Coloumb's constant and $k = \frac{1}{4\pi \varepsilon_0} = 8.99 * 10^9$. $\varepsilon_0 = 8.854 * 10^{-12}$ is the permitivity of free space, which does indicate that the value of $\varepsilon_0$ (and thus $k$) will change depending on the medium through which the field travels.

\section*{Dipoles}
A dipole is a area where two equal but opposite charge are in close proximity (separated by $d$, where $d$ is very small). When a dipole exists, the approximate electric field is \[ E \propto \frac{1}{r^n} \] where $n < 2$. We can also do vector addition with the following \[ r_1 + \frac{d}{2} = r_2 + \frac{d}{2} = r \] thus giving us the simplification for the electric field of a dipole system \[ E = \frac{kqd}{r^2} \]

\subsection*{Dipole Moment}
The dipole moment is written as \[ \vec{p} = q \cdot \vec{d} \] which is the charge multiplied by its displacement from the other dipole. Note that this simplifies our earlier dipole equation to \[ E = \frac{k\vec{p}}{r^2} \]

\section*{Charge Distribution}
\subsection*{Charge Density}
The charge density rho is equal to \[ \rho = \frac{Q}{V} \]

This gives \[ \dd E = \frac{k \rho}{r^2}\ \dd V \] and \[ E = \inint{\frac{k \rho}{r^2}}{V} \] Note that this is a vectorial integration, as r is changing in direction and size. This can only be easily done in symmetrical cases.

For a charge distributed evenly over a disk of radius $r$ \[ \rho = \frac{Q}{\pi r^2} \] and \[ \dd Q = \rho 2\pi r \dd r \]

The disk would create an electric field such that \[ \dd E = \frac{k \dd Q}{4\pi \varepsilon (r^2 + z^2)^{\frac{3}{2}}} \] and \[ E = \dint{0}{r}{\frac{\dd \rho\ 2\pi r\ z}{4\pi \varepsilon (r^2 + z^2)^{\frac{3}{2}}}}{r} \] which is \[ E = \frac{2\pi \rho z}{4\pi \varepsilon} \bigg[\frac{1}{2} * \frac{1}{\sqrt{r^2 + z^2}} \bigg] = \frac{\rho z}{4\varepsilon \sqrt{r^2 + z^2}} \]

\subsubsection*{Superposition}
Superposition is basically a simple concept: in electromagnetism, forces add. In a given system with $N$ charges, the force extered on one of them is \[ F = \sum_{i = 1}^{N} \frac{kq_0q_1}{r^2} \]

Given charge density, we can integrate this sum to find \[ F = \sum_{i = 1}^{N} \frac{kq\rho \Delta V}{r^2} \] given $N$ chunks of volume $\Delta V$ and charge $\rho \Delta V$. Technically, this formula is an approximation, but we can use integrals to determine the exact answer with \[ F = \dint{V}{}{\frac{kq\rho}{r^2}}{V} \]

\subsection*{Uniformly Distributed Charge}
For any point on an object with uniformly distributed charge, we can give it a width of $\dd y$ thus giving us
\begin{align*}
\dd E &= \frac{k \dd Q}{r_1^2} \\
      &= \frac{k \rho\ \dd y}{r^2 + y^2}
\end{align*}

So we have a general equation of the form \[ E = \dint{0}{L}{}{E} \] where $L$ is the length of the object. Note that this is a vectorial integral: as such, the direction of the field will always be perpendicular to the object (ie symmetrical along the horizontal axis).

Because this integral is symmetrical, we can integrate only along the axis, thus giving us
\begin{align*}
E &= \dint{-\frac{L}{2}}{\frac{L}{2}}{}{E}\\
  &= \dint{-\frac{L}{2}}{\frac{L}{2}}{\frac{k \rho r}{r_1^3}}{y}\\
  &= \dint{-\frac{L}{2}}{\frac{L}{2}}{\frac{k \rho r}{(r^2 + y^2)^{\frac{3}{2}}}}{y}\\
  &= \frac{k \rho L}{r\sqrt{r^2 + \frac{L^2}{4}}}
\end{align*}
Thus as $L \to \infty$, $E \propto \frac{1}{r}$

For charges uniformly distriuted on a ring, we have \[ \dd E = \frac{k \rho \dd L}{(r^\prime)^2} \]
or for any one direction \[ \dd E = \frac{k \rho \dd L}{(r^\prime)^2} \frac{z}{(r^\prime)^2} \] where $z$ is the direction along it's z-axis, so \[ E = \frac{k \rho z}{(r^\prime)^2} \inint{}{L} \] where $\inint{}{L}$ is the circumferance.

Thus gives us the general form of this equation \[ E_z = \frac{\rho rz}{2\varepsilon(r^2 + z^2)^{\frac{3}{2}}} \]

\section*{Electric Fields for Infinite Plains}
A plane can be written as a disk with an infinite radius. Then \[ \frac{1}{\sqrt{r^2 + z^2}} \to 0 \] so \[ E = \frac{\rho}{2\varepsilon} \] Note that this has no relation to $z$, and is constant with distance.

\section*{Electric Flux}
Electric flux is the \emph{flow} of some vectorial quantity (ie charge) through a given area. We can also think of it as (qualitatively) the number of field lines crossing a given area. It is given as the overlap between the \emph{amount of flow} and the \emph{given area}, thus giving us \[ \dd\Phi_E = E \dd A \] where$\dd A$ is the area of an open object. This gives us \[ \Phi_E = \inint{E}{A} \] or as a non-vectorial implementation \[ \Phi_E = EA\cos\theta \] where $\theta$ is the angle between the perpendicular vector to the object and the direction of the field lines.

By convention, we consider flux entering an object to be negative and flux leaving an object to be positive. This gives us an essential property: that for \emph{any} closed surface we have $\Phi_E = 0$, since as much flux enters the object as it does leave.

\subsection*{Gauss' Law}
If we have a closed surface with an electric field pointing in only one direction (example: a sphere of radius $r$ with a point charge $q$ enclosed in it's center), this gives us \[ \Phi_E = \inint{E}{A} = \inint{\frac{kq}{r^2}}{\pi r^2} = \frac{kq}{r^2} \frac{r^2}{\varepsilon_0} = \frac{1}{\varepsilon_0} q \]

If the surface is not spherical, we have the same equation: we can create a spherical region around the charge, segregate this area, and perform superposition to find our answer ($k 4\pi q + 0 = k 4\pi q$). Note that for more than one charge this gives us \[ \Phi_E = \frac{1}{\varepsilon_0} \sum_{i = 1}^N q_i \]

This is {\bf Gauss' Law}: that \emph{the electrix flux through a closed surface $S$ is equal to the total charge contained inside $S$ (divided by the permitivity of free space)}. Mathematically, we have \[ \Phi_E = \frac{q_{within}}{\varepsilon_0} \]

Note that we may refer to any closed surface as a {\bf gaussian surface}, since this law holds true in all such cases.

\subsubsection*{Using Gauss' Law}
Gauss' Law is indescribable useful in situations involving symmetry. The simplest case is in that of spherical symmetry.

Example: suppose we have a spherical distribution of charge with density $\rho(r)$ (ie the density is a function of the radius).

Enclose this distribution within a spherical surface $S$ of radius $r$. Gauss' Law gives \[ \Phi_E = \frac{q_{within}}{\varepsilon_0} \] Since the field is spherically symmetric, $E$ must point radially and be proportional to $r$. It can only depend on r, which means it is constant over any spherical surface. Thus we have the integral \[ \Phi_E = \frac{1}{\varepsilon_0} r^2 E(r) \]

The charge within $S$ is given by integrating the charge density over the volume of $S$ \[ q = \dint{V}{}{\rho(r)}{V} \] so \[ E(r) = \frac{\dint{V}{}{\rho(r)}{V}}{\varepsilon_0 r^2} = \frac{q(r)}{\varepsilon_0 r^2} \]

Note that outside of the sphere there is no charge ($\rho$ is constant), so we have
\begin{align*}
E(d) &= \frac{\rho d}{3\varepsilon_0} \\
     &= \frac{\rho r^3}{3\varepsilon_0 d^2}
\end{align*}

Or in other words: the electric field grows linearly with $d$ inside the sphere, but falls off inversely proportional to the square distance outside of the sphere (ie when $d > r$). Thus the external field is exactly that of a point charge \[ q = \frac{\rho r^3}{3\varepsilon_0} \]

\section*{Energy}
\subsection*{Work}
Work is a measure of the force exerted to move a charge from one location to another. It is given by \[ W = \inint{F}{r} \] where $F$ is the opposite of the force exerted upon that charge. This gives
\begin{align*}
W(r_0 \to r_1) &= \inint{F}{r}\\
               &= - \dint{r_1}{r_0}{\frac{kq_0q_1}{r^2}}{r}\\
               &= \frac{kq_0q_1}{r_0} - \frac{kq_0q_1}{r_1}
\end{align*}
An important fact to note is that the path travelled does not make a difference to the amount of work done; the only things which matter are the start and end positions.

\subsection*{Field Energy}
Field energy can be equated with pressure: it is the measure of force caused by an object's electric field per it's area. It is calculated by multiplying the electric field of a "hole" within the object by the charge density of the disk formed by creating this hole. This gives us \[ P = (E_{obj} - E_{disk})\rho \]

We can use this quantity to help us calculate work done with \[ \dd W = F \dd r = PA \dd r \]

In essence, this gives us "We had to put$\dd W = $ (energy density)$\dd V$ amount of work into the system \emph{to create} that field."

\subsection*{Potential Difference}
Potential difference is the measure of work which would have to be done to move a unit of charge from one location to another \emph{per unit of charge}. It is given by \[ \phi_{ba} = - \dint{a}{b}{E}{s} \] which can be derived by solving for the amount of work, then dividing out the charge (since work done is proportional to the charge). It can also be repesented by a $V$.

\subsubsection*{Potential}
The electric potential of an object is its potential difference with respect to some fixed point. Though this point is generally infinity, that is not always the case. When it is we have \[ \phi(r) = - \dint{\infty}{r}{E}{s} \]

We can also redefine potential difference in terms of potential (yes, this will be as painfully obvious as it sounds). Given $a$ and $b$, the potential difference between them is equal to \[ \phi_{ba} = \phi(b) - \phi(a) \]

If we suppose our field is created from a point charge $q$ at the origin we have \[ \phi(r) = - \dint{\infty}{r}{\frac{kq}{r^2}}{r} = \frac{kq}{r} \] which gives us a formula we can use to generalize this. Given superposition, we have the potential at some point $P$ from $N$ contributing charges as \[ \phi(P) = \sum_{i = 1}^N \frac{kq_i}{r_i} \] this can then be further generalized to account for continuous distributions with \[ \phi(P) = \dint{V}{}{\frac{k \rho}{r}}{V} \]

Obviously, these equations are only useful if we \emph{can} set the reference point to infinity. This, in turn, is only possible if the charge distribution is of a finite size. If the distribution is infinitely large, we will find the equation to diverge as the reference point approaches infinity. In such a case, simply pick a different reference point.

\subsection*{Charged Conductors}
Since we can also express work as \[ W = -q(\phi(b) - \phi(a)) \] we can see that no work is required to move a charge between two points at the same potential. We also know that the electric potential is constant everywhere on the surface of a charged conductor in equilibrium, and the the electric potential is constant everywhere inside a conductor and is equal to the value at its surface.

Any surface on which all points are at the same potential is called an {\bf equipotential surface}. The potential difference of any two points on said surface is zero, no work is required to move charge at a constant speed on the surface, and the electric field is always perpendicular to this surface.

\section*{Electrostatic Equilibrium}
A conductor is in electrostatic equilibrium when no net motion of charge occurs within it. If this is true, then we also know the following
\begin{itemize}
\item The electric field is zero everywhere inside the conductor.
\item Any excess charge on an isolated conductor resides entirely on its surface.
\item The electric field outside a charged conductor is perpendicular to its surface.
\item For any non-spherical (ie misshapen) conductor, charge accumulates on its "sharpest" points.
\end{itemize}

\section*{Electric Current}
Current is defined as the rate at which charge flows through a surface. Mathematically, we have \[ I = \frac{q}{t} \] which is measured in {\bf amperes} $A = \frac{C}{s}$.

The direction of current is defined to be the direction at which a \emph{positive} charge would flow (note that this is opposite the direction of electron movement!). We refer to moving charge through a conductor as a {\bf mobile charge carrier}.

Electrons flow in the opposite direction of an electric field, as given by the "opposites attract" rule-of-thumb. As a charged particle moves through a conductor, it collides with atoms/molecules in the conductor, thus giving it an erratic movement pattern through the conductor.

Although particles follow an erratic path, we can find its average speed (called {\bf drift speed}) as \[ v_d = \frac{\Delta x}{t} \] This is particularly useful when combined with a general equation for the amount of charge passing through a wire given by \[ \Delta Q = n A \Delta x q \] where we have the number $n$ of charge carriers, the cross-secitonal area $A$ of the wire, a small slice $\Delta x$ along the wire, and the charge $q$ of each particle. We can combine these equations to give \[ \Delta Q = n A v_d t q \]

By dividing both sides of this equation by $t$, we get \[ I = n A v_d q \] or \[ v_d = \frac{I}{n q A} \]

An electric field exists within a conductor if and only if it has a non-zero current.

\section*{Capacitors}
A capacitor is a device which can store charge for a short period of time. It usually consists of two parallel conducting plates separated by a very small distane. One plate is connected to a positive voltage source, the other to a negative one, and the electrons from the positive plate are thus pulled onto the other plate.

The transfer of charge stops when the potential difference across the plates is equal to the potential difference of the battery (or other voltage source). Once the capacitor is charged, it acts as storage for charge and energy.

The {\bf capactiance} $C$ of a capacitor is the ratio of charge on either plate to potential difference between the plates \[ C = \frac{q}{V} \] It is measured in Farads (Coulombs per Volt), but we will never approach this value. One Farad is an obscenely large capacitance: more often we will see values between 1 pF ($10^{-12}$ F) and 1 $\mu$F ($10^{-6}$ F).

\subsection*{Parallel Plate Capacitors}
A parallel plate capacitor is a special class of capacitor which we will frequently deal with. For one such capacitor, we can also find the capacitance by \[ C = \frac{A \varepsilon_0}{d} \] where we have the area of a plate, the distance between the paltes, and the permitivity of free space.

In a circuit diagram, a capacitor is labelled as a drawing of parallel plates, with a curved line sometimes representing the negative plate.

\subsection*{Combinations of Capacitors}
Capacitors act in the opposite fashion as do resistors: to find the equivalent capacitance of parallel capacitors we simply add the capacitances, but to find the equivalent of a series of capacitances we must take the inverse of the sum of inverses.

\subsection*{Energy Stored in Charges Capacitors}
An uncharged capacitor contains no energy, but as we charge it an electric field is generated betweent the plates and the system gains energy. The potential energy given by this field is equivalent to the {\bf internal energy} $U$ of the capacitor, thus we have \[ U = qV \] Given the relationship between voltage, charge, and capacitance, we can find the algebraic form of this equation to be \[ \dd U = \frac{q}{C} \dd q \]

Since $C$ remains constant as $U$ and $q$ change, we can integrate this to find the energy as a function of the charge. This gives us \[ U = \frac{q^2}{2C} = \frac{CV^2}{2} = \frac{qV}{2} \]

\subsection*{Dielectrics}
A dielectric is any type of insulating material. For a capacitor with no dielectric, the equations above hold (ie $V = \frac{q}{C}$). When we insert a dielectric between the plates, the "voltage drop" is reduced by a scale factor $\kappa > 1$ (the {\bf dielectric constant}).

Since the charge on teh capacitor will (obviously) not change when we introduce a dielectric, the capacitance must change to the value \[ C = \frac{\kappa q}{V} \] We also rewrite the parallel plate equation as \[ C = \frac{\kappa A \varepsilon_0}{d} \]

For any given plate separation, there is a maximum electric field which can be produced through the dielectric before it begins to conduct. This maximum field is called the {\bf dielectric strength}. If the field exceeds the dielectric strength, the capacitor will short circuit. Generally, we don't worry much about this, as the dielectric strength of air is approximately $3$ x $10^6$ V/m.

\section*{Resistance}
Resistance is the ratio of voltage to current \[ R = \frac{V}{I} \] It is measured in {\bf ohms}  which are given by $\Omega = \frac{V}{A}$.

Resistance is a measure of the "difficulty" of a charged particle passing through a material; essentially, the number of internal collisions a particle will see. It also describes the amount of heat generated by the material, since an increased number of collisions per second will increase the amount of heat generated "by" the material.

\subsection*{Ohm's Law}
Resistance that remains constant over a wide range of applied voltage differences causes voltage drops to be linearly dependant on current \[ V = IR \] Most materials (refered to as {\bf ohmic}) obey this principle, though some materials (such as semi-conductors) do not.

\subsection*{Resistivity}
We can describe resistance in trms of the geometric properties and composition of the conductor with \[ R = \frac{L \rho}{A} \] where we have the legnth, area, and resistivity of the conductor.

\subsubsection*{Temperature Variance}
For most metals, resistivity changes approximately linearly with termperature \[ \rho = \rho_0 \bigl[ 1 + \alpha(T - T_0) \bigl] \] where $\alpha$ is the {\bf temperature coefficient of resistivity}.

If a wire is of constant area and length, we can write \[ R = R_0 \bigl[ 1 + \alpha (T - T_0) \bigl] \]

\section*{Power}
Power is a measure of work exerted over a period of time \[ \Power = \frac{W}{t} \]

Given various ways of manipulating this, we have \[ \Power = IV = \frac{V^2}{R} = I^2R \] where any definition of power containing resistivity applies only to ohmic materials.

\section*{Magentism}
\subsection*{Magnets}
Magnetism, being a component of electromagnetism, thusly acts quite similarly to electricity: like {\bf poles} repel each other and unlike ones attract, much like charges. Instead of positive and negative, we have north and south poles. Note: magnets must be assymetrical to have two poles, generally in the shape of a bar.

Like the electric field, magnets have the {\bf magnetic field} $H$. This field points away from the north pole and towards the south pole. We also define the magnetic flux density given by $B = \mu_0 H$. We tend to refer to the magnetic flux density $B$ as the magnetic field.

Unlike electricity, there are no monopoles in magnetism: magnets are always dipoles.

\subsection*{Magnetic Fields}
A moving charged particle creates a magnetic field. This is given by \[ B = \frac{\mu_0}{4\pi} \frac{qv\sin\theta}{r^2} \]

When moving through a magnetic field, a charged particle also experiences a force. The force is at its manimum when the charge moves \emph{perpendicular} to the $B$-field lines, and becomes zero when the two are parallel. This force is given by \[ F = qvB \sin\theta \]

The units for a magnetic field is the {\bf Tesla} T. It is also sometimes measured in {\bf Webers} Wb where $1$ T $= 1$ Wb/m$^2$, where a weber is the unit of magnetic flux (not yet introduced). To give an idea of perspective, the magnetic field of the Earth at the equator is approximately $31 \mu$T.

\subsubsection*{Magnetic Field of the Earth}
Currently, the magnetic north pole corresponds to the southern geographic pole, and vice versa. The magnetic axis is not aligned with the spin axis.

This field is generated by electric currents in the liquid outer core of the planet, which in turn arise from the Earth's rotation. It flips its polarity every now and again (ie at random intervals over time), and has flipped 171 times in 76 million years.

\subsubsection*{Biot-Savart Law}
The Biot-Savart Law gives the relationship between a constant current and the magnetic flux density it generates with \[ \dd B = \bigg( \frac{\mu_0}{4\pi} \bigg) \frac{I \dd l \times \vec{r}}{r^2} \] where $\vec{r}$ is the directional vector (i.e. it could be replaced by $\sin\theta$, given the proper manipulations to the equation) and the {\bf permeability of free space} $\mu_0 = 4\pi$ x $10^{-7}$ T m/A.

Note that the direction never changes with this integral (ie $\intone{B}$), only the amplitude changes. This allows us to integrate it with a minimum of effort.

We can find the force exerted by this field as $\dd F = I_1 \dd l_1 \bigl( \frac{\mu_0}{4\pi} \bigl) \frac{I_0 \dd l_0 \times \vec{r}}{r^2}$, or, more concisely, \[ \dd F = \dd B I \dd l \]

\subsubsection*{Current Density}
If we are given the current density $J$ of a system, we can change our integral slightly by using our current density instead of our current to give us \[ B = \frac{\mu_0}{4\pi} \inint{\frac{J \times \vec{r}}{r^2}}{V} \] since $I \dd l = J \dd s \dd l = J \dd V$ and $I = \inint{J}{S}$.

This equation is only useful if the current density is given in place of the current.

Note that we also have $J = qv$, for an object with some total charge and velocity. This can be useful in sistuations where we have charged objects with no current.

\subsubsection*{Ampere's Law}
The $B$-fields around a wire will exist concentrically and will point clock-wise (assuming we look in the direction the current is traveling. We can determine its strength with \[ B = \frac{\mu_0 I}{2\pi r} \] Note that technically $2\pi r$ is given by $\intone{l}$, but we can virtually always make this simplification.

Note that its strength decreases directly proportionally to distance.

\subsubsection*{Law for Circuits}
For a circuit with multiple components, we can use \[ \inint{B}{l} = \mu_0 I \] Note that we can derive the general form of Ampere's law from this given $B$ is constant and $\intone{l} = 2\pi r$.

We can also use this law to determine the effect of two parallel wires upon each other per unit of length as \[ \frac{F}{l} = \frac{\mu_0 I_0 I_1}{2\pi r} \] This force is attractive if the currents are in the same direction.

\subsection*{Magnetic Force on a Current-Carrying Conductor}
A current-carrying wire also experiences force when passing through a magnetic field. This force is given by \[ \dd F = I \dd l \times\dd B \] where we have the magnetic flux density, the current within the conductor, and the length of said conductor. The direction of the force will be perpendicular to the plane formed by the two forementioned vectors.

Example: for a square circuit of current $I_2$ a distance $R$ away from an infinite line of current $I_1$, where the dimensions of the circuit are given by $a,b$, the force on the circuit from the wire is given by $F = \frac{\mu_0 I_1 I_2}{2\pi R} b (\vec{r}) + \frac{\mu_0 I_1 I_2}{2\pi (R + a)} b (\vec{-r})$.

\subsection*{Motion of a Charged Particle in a $B$-Field}
A charged particle will always move in a circular path through a $B$-field since velocity, force, and $B$-field direction must be perpendicular (see: right-hand rule).

Since the particle is in circular motion, there must be a centripetal force on the particle. This is given by \[ F = q(v \times B) = \frac{mv^2}{r} \] The {\bf radius of curvature} of the path is then \[ r = \frac{mv}{qB} \]

This gives us {\bf Lorenz' Forces Equation}, which is the force acting on a particle in magnetic and/or electric fields given by \[ F = qE + q(v \times B) \]

\subsection*{Magnetic Flux}
Much the same as electric flux, magnetic flux is given by \[ \Phi_B = n\inint{B}{A} \] where $n$ is the number of loops. As previsously mentioned, it is measured in Webers. Note that it is found by multiplying the magnetic flux \emph{density} by the area over which it exists. It is for this reason that we prefer to use magnetic flux density over magnetic field strength.

We also have the {\bf flux linkage}, which is equal to \[ \land = n\Phi_B \] Note that in this case the $n$ is the loops in the other circuit, so we have (more generally) \[ \land = n_0n_1 \inint{B}{A} \]

\subsection*{Inductance}
We can also define magnetic flux linkage as \[ \land = n_0n_1\phi_B = LI \] where $L$ is the inductance proportionality between the flux surface and the current generating a field. Thus we can find inductance with \[ L = \frac{n_0n_1 \inint{B}{A}}{I} \] Note that any circuit will also have a {\bf self-inductance} which we can solve with \[ L_{self} = \frac{\Phi_B}{I} \]

\subsection*{Time-Varying Fields}
If the flux linkage of a field varies over time, a voltage is induced. By {\bf Lenz' Law}, the induced voltage is created to oppose the change which is happening. Basically, Lenz' Law is a statement of the conservation of energy, since the induced voltage is simply a measure of potential energy, and describes the relationship between $\dd l$ and $\dd S$. The value of the induced voltage is given by {\bf Faraday's Law} \[ V = -\frac{\dd \land}{\dd t} \] We can also see that $V = \inint{E}{l} = -\frac{\dd}{\dd t} n_0n_1 \inint{B}{A}$.

Note that $V = \inint{E}{l}$ instead of what we would expect ($V = -\inint{E}{l}$) because this type of phenomena acts like a battery or other source... and current flows "backwards" in a voltage source (i.e. from negative \emph{to} positive).

It is important to realise that induced voltage is non-conservative. As such, we can see that in any circuit with a current or voltage induced by a magnetic field will not obey Kirchoff's Law. In such a circuit, we could measure (with voltemeters) the voltage of various aspects of our circuit and find a non-zero total. This total will be equal to the induced voltage.

We can have various aspects of our circuit changing with respect to time:
\begin{itemize}
\item If $B$ is changing, we have {\bf transformer EMF}.
\item If $S$ is changing, our circuit is expanding or contracting and we have {\bf motional EMF}.
\item If $B \cdot \dd S$ is changing, our direction of current is changing and we have motional EMF.
\item If all of the above are changing simultaneously, we have both types of EMF.
\end{itemize}

\subsubsection*{Transformer EMF}
If we have a changing $B(t)$, then we can find the induced voltage with \[ V = -n_0n_1 \inint{\frac{\dd B}{\dd t}}{S} \]

\subsubsection*{Motional EMF}
If we have motional EMF the induced voltage is given by \[ V = - \frac{\dd \land}{\dd t} = - \frac{\dd}{\dd t}\ n_0n_1 \inint{B}{S} \]

In this case, the circuit is moving, so either the enclosed surface is changing, or $B \cdot \dd S$ is changing. Thus we have other ways to solve for either of these cases, and can find the induced voltage with \[ V = \inint{\vec{u} \times B}{l} \] where $\vec{u}$ is the velocity vector. In this way, we also have \[ V = - B_0 u_0 h \] where $u_0 t$ is the distance travelled by a non-fixed rod of length $h$.

Alternatively, we can use \[ \inint{B}{S} = \inint{B \times \vec{n}}{S} \] where $\vec{n}$ is the normal vector of $\dd S$.

This gives us an important relationship to be used in cases where there exists some circuit with magnetic flux \[ IR = \bigg|\frac{\dd \phi_B}{\dd t}\bigg| \]

\subsection*{Magnetic Field $H$ vs Magnetic Flux Density $B$}
Recall that magnetic flux $\phi_B = \inint{B}{S}$. From this we see that $B$ is the magnetic flux density. The magnetic flux density and magnetic field (intensity) are related to each other by $B = \mu_0 H$.

Since $\dd B = \mu_0 \dd H$, we can write some of our common formulas in the form $\inint{B}{l} = \mu_0 I \implies \inint{H}{l} = I$.

\subsection*{Magnetic Dipoles}
Any current-carrying loop is a magnetic dipole. We define the {\bf magnetic moment} as $\vec{m} = I \cdot A$, with the direction given by the right-hand rule.

We can see that an atom is a (complex, multi-dimensional) magnetic dipole, and so can understand that strong fields can be created by carefully arranging enough atoms in a specific configuration.

On a macrospcopic level, we can find the {\bf total magnetism} of an object as $M = \sum \vec{m}$.

\subsection*{Permeability}
We can find the permeability of an object given by \[ \mu = \mu_0 \mu_r \] where $\mu_r$ is an object's maximum magnetisation amount or {\bf relative permeability}. Example: $\mu_r$ for iron is 10.

\section*{Drift Velocity}
Drift velocity is the average velocity that a particle attains due to an electromagnetic field. It can be found either with \[ J = \rho v \] where $J$ is the current density and $\rho$ is the charge density, or \[ v = \mu E \] where $\mu$ is the electron mobility and $E$ is the electric field.

\section*{Maxwell's Equations}
We have {\bf Gauss' Law} \[ \inint{D}{S} = Q \implies \nabla D = \rho \] and the {\bf magnetic form} of Gauss' Law, which implies that there exist no magnetic monopoles \[ \inint{(\nabla \times B)}{l} = 0 \implies \nabla B = 0 \] and {\bf Faraday's Law} \[ \inint{E}{l} = -\frac{\dd \phi}{\dd t} \implies \nabla \times E = =\frac{\dd B}{\dd t} \] and {\bf Ampere's Law} \[ \inint{H}{l} = I \implies \nabla \times H = J + \frac{\dd D}{\dd t} \]

In Maxwell's form and in a region of space where $J = \rho = 0$, the first two and the last two equations are coupled.
\end{document}
