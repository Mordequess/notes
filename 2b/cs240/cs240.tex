\documentclass[12pt]{article}
\usepackage{amsmath,amssymb,parskip,custom}
\usepackage[margin=1in]{geometry}
\begin{document}

\title{CS 240 --- Data Structures and Data Management}
\author{Kevin James}
\date{\vspace{-2ex}Spring 2014}
\maketitle\HRule

\section{Algorithms}
An {\bf algorithm} is a step-by-step process for carrying out a set of operations given an arbitrary problem instance. An algorithm {\bf solves} a problem if, for every instance of the problem, the algorithm finds a valid solution in finite time.

A {\bf program} is an implementation of an algorithm using a specified programming language.

For each problem we can have several algorithms and for each algorithm we can have several programs (implementations).

In practice, given a problem:
\begin{enumerate}
\item Design an algorithm.
\item Assess the correctness of that algorithm.
\item If the algorithm is acceptable, implement it. Otherwise, return to step 1.
\end{enumerate}

When determining the efficiency of algorithms, we tend to be primarily concerned with either the runtime or the memory requirements. In this course, we will focus mostly on the runtime.

To perform runtime analysis, we may simply mplement the algorithm and use some method to determine the end-to-end time of the program. Unfortunately, this approach has many variables: test system, programming language, programmer skill, compiler choice, input selection, \dots



\end{document}
