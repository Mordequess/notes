\documentclass[12pt]{article}
\usepackage{amsmath,amssymb,parskip,custom}
\usepackage[margin=1in]{geometry}

\begin{document}

\title{ENGL 119 --- Communications in Math and Computer Science}
\author{Kevin James}
\date{\vspace{-2ex}Spring 2014}
\maketitle\HRule

\section{Communication}
This class will primarily focus on
\begin{itemize}
\item selecting the right information
\item choosing the right words
\item organizing and presenting this information
\end{itemize}

When communicating, it is important to take into account the {\bf audience}, {\bf purpose}, and {\bf context}. The audience and purpose are mostly self-explanatory, and the context deals with everything else: the surrounding circumstances, situation, medium of communication, et cetera.

Failures in communication are often the result of having poor coherence, disregarding conventions, or being insensitive to professional expectations.

\section{Interview Skills}
The three skills to develop in order to increase your chances of cussess in an interview are:
\begin{itemize}
\item the \emph{content} of your responses
\item the clarity of \emph{expression} of your responses
\item your \emph{manner}
\end{itemize}

When dealing with content, you should be prepared to speak about your past performance in order to prove the claims you are making for yourself. In addition, you should try to anticipate the kinds of questions you may be asked by researching the company and the job position.

More specifically, you should mention examples of how you have contributed to a project, worked under pressire, demonstrated creativity, resolved conflict or solved problems, organized work, and prioritized or delegated responsibilities.

Without these examples, your answers will sound hesitant and insubstantial and will lack conviction. Without preparation and a focus for your interview, you may waste too much time rambling before coming to your point.

When asked a question, it is generally a good idea to state your point directly, follow that with some substantiation, and then open up a discussion based on the central point of your answer.

\section{Resume Writing}
There are two main features of resume writing: persuasiveness and useability. In other words, you want your resume to have a lot of information on who you are, though it must also be concise and easy to scan.

Within thirty seconds, a job recruiter glancing at your resume should be able to see your skill set and your work experience.

You should consider organization, layout, and design when creating your resume, and should also ensure an adequate (read: immeasurable) amount of proofreading has been done.



\end{document}
