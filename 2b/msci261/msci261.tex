\documentclass[12pt]{article}
\usepackage{amsmath,amssymb,parskip,custom}
\usepackage[margin=1in]{geometry}

\begin{document}

\title{MSCI 261 --- Engineering Economics}
\author{Kevin James}
\date{\vspace{-2ex}Spring 2014}
\maketitle\HRule

\section{Engineering Economic Analysis}
{\bf Engineering economic analysis} can be defined as the economic analysis of costs, benefits, and revenues occuring over time. This course develops the tools used to solve these problems in an engineering setting.

This course will focus specifically on {\bf discrete time finances}, which is a simplified version of actual finance and does not involve calculus.

\section{Time Value of Money}
All money has a {\bf time value} associated with this. This value represents the returns one could make on an investment of some amount of money at time $x$ versus a potentially different amount at time $y$.

For example, when computing the difference between receiving $\$1000$ right now ($x$) versus $\$1300$ in three years ($y$), we determine what interest rates we could receive. So for option $x$, if we can guarantee investment returns of $10\%$, then in three years we would have $\$1000 + \$100 + \$110 + \$121 = \$1331$. Therefore, $x$ is the better option, despite being a statically lower value.

Note that `interest rates' do not always refer to, say, the interest received by placing one's money in a savings account. If you are starting a company, this determination becomes something different: what would the value of your company be in three years if you had $\$1000$ now versus $\$1300$ in three years?

\end{document}
