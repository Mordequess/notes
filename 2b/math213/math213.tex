\documentclass[12pt]{article}
\usepackage{amsmath,amssymb,parskip,custom}
\usepackage[margin=1in]{geometry}

\begin{document}

\title{MATH 213 --- Advanced Mathematics for Software Engineers}
\author{Kevin James}
\date{\vspace{-2ex}Spring 2014}
\maketitle\HRule

\section{Differential Equations}
{\bf Differential equations} are equations involving derivatives with respect to some independant variable. For example, Newton's Law states \[ M \ddot x = F \] or \[ M \dderiv{x}{t} = F \]

In the {\bf classical approach}, we suppose $f$ is given as a function of time, and we solve for the dependant variable with respect to the independant one. For example, \[ F(x) \function x(t) \]

The {\bf systems approach} has less of an emphasis on the response to a specific input and deals more with the overall relationships between the function and between the individual dependant variables.

\subsection{Examples}
The population of an organism (given abundant resources) or the growth of an economy (if the economy were to grow at a constant percentage rate) can be modelled as:
\begin{align*}
\dot x &= ax\\
\frac{\dd x}{x} &= a \dd t\\
\int \frac{\dd x}{x} &= \int a \dd t\\
\ln x + C_1 &= at + C_2\\
\ln x &= at + C_3\\
x(t) &= e^{at + C_3}\\
x(t) &= e^{at} \times e^{C_3}\\
x(t) &= C_4 \times e^{at}\\
x(t) &= x(0) \times e^{at}
\end{align*}
where the value of $x(0)$ is called the {\bf initial condition}.

Given that this function assumes that the population growth is not limited by resources, etc., it is not very useful in the real world. More likely, we would find for large populations a limit of some sort must be included. For example, the {\bf logistic equation} is modelled as:
\begin{align*}
\dot x &= ax - bx^2\\
\frac{\dd x}{ax - bx^2} &= \dd t\\
\int \frac{\dd x}{ax - bx^2} &= \int \dd t\\
\int \frac{\dd x}{x(a - bx)} &= \int \dd t\\
\int \dd x (\frac{A}{x} + \frac{B}{a - bx}) &= \int \dd t\\
\int \dd x (\frac{1}{ax} + \frac{B}{a - bx}) &= \int \dd t\\
\int \dd x (\frac{1}{ax} + \frac{b}{a(a - bx)}) &= \int \dd t\\
\int \frac{\dd x}{ax} + \int \frac{b \dd x}{a(a - bx)} &= \int \dd t\\
\frac{\ln x}{a} + \frac{b}{a}\frac{-1}{b}\ln(a - bx) &= t + C_0\\
\frac{1}{a} \bigg(\ln\frac{x}{a - bx}\bigg) &= t + C_0\\
\frac{x}{a - bx} &= e^{at + aC_0}\\
x &= C_1 e^{at} (a - bx)\\
x &= \frac{aC_1 e^{at}}{1 + bC_1e^{at}}\\
x &= \frac{a}{b} \bigg(\frac{1}{1 + C_2 e^{-at}}\bigg)
\end{align*}
Where $C_2 = \frac{1}{bC_1}$. In this case, the population will ``level out'' at $ax = bx^2$ (i.e.\ have an asymptote). The solution to this specific {\bf DE (differential equation)} is called the {\bf logistic curve}.

\subsection{Partial Differential Equations}
{\bf PDEs (partial differential equations)} arise when there is more than one independant variable.

If we were to model the vibration of a string, we would use a PDE.\@ Assuming there is no length-wise vibration (i.e.\ horizontal motion), that the string has constant tension and mass per unit length, and that we are only considering small transverse displacements, we could write Newton's equation $F = ma$ as \[ F = (\rho\Delta x) \times \dderiv{y}{t} \]

Since all forces on this string are tension, we have \[ F = T\sin\theta_1 - T\sin\theta_2 \]

Given that we have small displacements (which leads to small angles $\theta_1$ and $\theta_2$), we can replace all instances of $\sin$ with $\tan$. Finally, this gives us \[ F \approx T\deriv{y}{x} \bigg|_{x+\frac{\Delta x}{2}} - T\deriv{y}{x} \bigg|_{x-\frac{\Delta x}{2}} \]

so therefore \[ \rho\dderiv{y}{t} = \frac{T}{\Delta x} \bigg(\deriv{y}{x} \bigg|_{x+\frac{\Delta x}{2}} - \deriv{y}{x} \bigg|_{x-\frac{\Delta x}{2}}\bigg) \]

As $\Delta x$ approaches zero, we see that \[ \dderiv{y}{t} = \frac{T}{\rho} \dderiv{y}{X} \]

We will not be solving PDEs in this course, but this equation is solveable to give us \[ y = A\sin k\bigg(x - t\sqrt{\frac{T}{\rho}}\bigg) \]

\subsubsection{Boundary Conditions}
This equation should additionally have several {\bf boundary conditions} which represent the ends of the string being fixed in place.

\begin{align*}
y(0, t) &= 0\\
y(L, t) &= 0
\end{align*}

We use these equations as well as the equation for a standing wave to get \[ y = A_+ \sin k \bigg( x - t\sqrt{\frac{T}{p}} \bigg) + A_- \sin k \bigg( L - t\sqrt{\frac{T}{\rho}} \bigg) \]

Since $y(0, t) = 0$,
\begin{align*}
0 &= A_+ \sin k \bigg( 0 - t\sqrt{\frac{T}{p}} \bigg) + A_- \sin k \bigg( 0 - t\sqrt{\frac{T}{\rho}} \bigg)\\
A_+ &= A_- = A
\end{align*}

and since $y(L, t) = 0$,
\begin{align*}
y &= A_+ \sin k \bigg( L - t\sqrt{\frac{T}{p}} \bigg) + A_- \sin k \bigg( L - t\sqrt{\frac{T}{\rho}} \bigg)\\
KL &= \pm n\pi
\end{align*}

Because waves can only have certain frequences we have
\begin{align*}
K \sqrt\frac{T}{\rho} &= \pm \frac{n\pi}{L} \sqrt\frac{T}{\rho}\\
f &= \pm \frac{n}{2L} \sqrt\frac{T}{\rho}
\end{align*}
where $n = 1$ implies a fundamental frequency and $n \geq 2$ implies a harmonic frequency.

\end{document}
